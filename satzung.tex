\documentclass[a4paper, 12pt]{scrartcl}
\usepackage{fontspec}
\usepackage{polyglossia}
\setmainlanguage[spelling=new,latesthyphen=true]{german}
\pagestyle{plain}
\newcommand\verein{Blinkstube}
\title{Satzung}
\subtitle{\verein}
\author{}
\date{18. Februar 2019}

\renewcommand*\thesection{\S~\arabic{section}}
\KOMAoptions{toc=flat}
\begin{document}
\maketitle
\sffamily
%\tableofcontents

%\newpage
\section{Name, Sitz, Geschäftsjahr}
\begin{enumerate}
  \item Der Verein führt den Namen \verein.
  \item Er hat seinen Sitz in Stuttgart.
  \item Er wird in das Vereinsregister beim Registergericht Stuttgart eingetragen. Nach Eintragung führt er den Zusatz „e. V.“.
  \item Geschäftsjahr ist das Kalenderjahr.
\end{enumerate}

\section{Zweck des Vereins}
\begin{enumerate}
  \item Zweck des Vereins ist die Förderung der Erziehung, Volks- und Berufsbildung einschließlich der Studentenhilfe.
  \item Besonders im Fokus steht dabei die Bildung und Fortbildung von Kindern, Jugendlichen, und Erwachsenen im Bereich digitale Medien, Software-, und Hardwareentwicklung, sowie die kulturellen und gesellschaftlichen Entwicklungen, die sich daraus ergeben.
  \item Der Satzungszweck wird unter anderem umgesetzt durch
  \begin{enumerate}
    \item Die Bereitstellung von Räumen für Vorträge und Bildungsveranstaltungen.
    \item Die Organisation von Vorträgen und Bildungsveranstaltungen zu moderner Informationstechnologien sowie dem Internet.
    \item Jugendarbeit und Erwachsenenbildung im Bereich Medienkompetenz. Beispielsweise Schulungen zum verantwortungsvollen Umgang mit neuen Medien in Zusammenarbeit mit öffentlichen sowie privaten Bildungseinrichtungen.
    \end{enumerate}
  \item Der Verein verfolgt ausschließlich und unmittelbar gemeinnützige Zwecke im Sinne des Abschnitts „steuerbegünstigte Zwecke“ der Abgabenordnung (§ 51 ff AO).
\end{enumerate}

\section{Selbstlosigkeit}

\begin{enumerate}
  \item Der Verein ist selbstlos tätig, er verfolgt nicht in erster Linie eigenwirtschaftlicheZwecke.
  \item Mittel des Vereins dürfen nur für satzungsgemäße Zwecke verwendet werden. Die Mitglieder erhalten keine Zuwendungen aus Mitteln des Vereins.
  \item Es darf keine Person durch Ausgaben, die dem Zweck der Körperschaft fremdsind, oder durch unverhältnismäßig hohe Vergütungen begünstigt werden.
\end{enumerate}

\section{Mitgliedschaft}
\begin{enumerate}
  \item Der Verein besteht aus natürlichen und juristischen Personen.
  \item Ordentliche Mitglieder können nur natürliche Personen werden. Fördermitglieder können natürliche Personen und juristische Personen jedweder Rechtsform werden.
  \item Fördermitglieder unterstützen die Zwecke des Vereins vor allem durch Zuwendungen, insbesondere finanzieller Art. Sie besitzen kein Stimmrecht auf der Mitgliederversammlung, haben jedoch ein Informationsrecht zu allen Belangen des Vereins.
  \item Der Vorstand entscheidet auf schriftlichen oder in Textform gestellten Antrag des Antragstellers über die Aufnahme. Der Beschluss wird dem Antragsteller schriftlich oder in Textform mitgeteilt.
  \item Der Mitgliedsstatus natürlicher Personen lässt sich auf Antrag von einer Fördermitgliedschaft in eine ordentliche Mitgliedschaft wandeln und umgekehrt. Überden Antrag entscheidet der Vorstand.
  \item Die Mitgliedschaft endet
  \begin{enumerate}
    \item bei juristischen Personen mit deren Löschung.
    \item bei natürlichen Personen mit deren Tod.
    \item nach schriftlicher Kündigung mit einer Frist von drei Wochen zum Ende des Monats. Die Kündigung muss im Rahmen dieser Frist schriftlich oder in Textform beim Vorstand eingegangen sein.
    \item bei Mitgliedern, die sich nach schriftlicher Mahnung oder Mahnung in Textform mit mehr als vier Monatsbeiträgen im Verzug befinden.
    \item bei Ausschluss des Mitglieds.
  \end{enumerate}
\end{enumerate}

\section{Mitgliedsbeiträge}
Die Mitglieder entrichten Mitgliedsbeiträge nach der durch die Mitgliederversammlung festgelegten Beitragsordnung.

\section{Vereinsorgane}
Die Organe des Vereins sind:
\begin{enumerate}
  \item die Mitgliederversammlung
  \item der Vorstand
\end{enumerate}

\section{Die Mitgliederversammlung}
\begin{enumerate}
  \item Die Mitgliederversammlung besteht aus den Mitgliedern des Vereins, deren Mitgliedschaft nicht ruht.
  \item Stimmberechtigt ist jedes anwesende ordentliche Mitglied dessen Mitgliedschaft nicht ruht.
  \item Die ordentliche Mitgliederversammlung ist mindestens ein Mal pro Jahr vom Vorstand mit einer mindestens 14-tägigen Frist einzuberufen.
  \item Die Einladung erfolgt in schriftlicher oder elektronischer Form.
  \item Die außerordentliche Mitgliederversammlung kann vom Vorstand einberufen werden.
  \item Der Vorstand hat unverzüglich eine außerordentliche Mitgliederversammlung einzuberufen, wenn 20\% der Vereinsmitglieder die Einberufung schriftlich fordern.
  \item Zusätzliche Tagesordnungspunkte, die auf der Mitgliederversammlung behandelt werden sollen, müssen eine Woche vor dieser beim Vorstand eingegangen sein. Dieser hat die weiteren Tagesordnungspunkte unverzüglich in den vereinsinternen Medien zu veröffentlichen.
  \item Eine Vertretung eines ordentlichen Mitgliedes durch ein anderes ordentliches Mitglied ist möglich, wenn die Vertretungsbefugnis schriftlich nachgewiesen wird und der Versammlungsleitung vor Beginn der Veranstaltung bekannt gegeben wird. Jedes ordentliche Mitglied darf höchstens ein anderes ordentliches Mitglied vertreten.
  \item Die Mitgliederversammlung ist beschlussfähig, wenn mindestens 20\% der beschlussfähigen Mitglieder anwesend ist.
  \item Ist die Mitgliederversammlung nicht beschlussfähig, wird eine Wiederholungsversammlung einberufen, die in jedem Falle beschlussfähig ist.
  \item Abstimmungen müssen geheim erfolgen, wenn mindestens ein Mitglied dies fordert.
\end{enumerate}

\section{Zuständigkeit der Mitgliederversammlung}
Die Mitgliederversammlung
\begin{enumerate}
  \item wählt den Vorstand. Die Wahl jeder zu besetzenden Stelle erfolgt durch Zustimmungswahl. Jedes ordentliche Mitglied kann für eine beliebige Anzahl von Kandidaten stimmen. Die Kandidatin mit den meisten Stimmen ist gewählt. Bei Gleichstand kommt es zur Stichwahl. Werden mehrere gleiche Stellen besetzt, werden die Wahlgänge kombiniert. In diesem Fall sind die Kandidaten mit den meisten Stimmen gewählt.
  \item prüft und genehmigt die Jahresabschlussrechnung des Vorstands und erteilt die Entlastung des Vorstandes.
  \item entscheidet in allen Fällen, in denen nicht die Zuständigkeit eines anderen Organs bestimmt ist.
  \item  wählt aus ihren Reihen einen Protokollführer, der den Ablauf der Mitgliederversammlung schriftlich oder in Textform protokolliert.
\end{enumerate}

\section{Vorstand}
\begin{enumerate}
  \item Der Vorstand gemäß § 26 BGB besteht aus drei Personen, von denen eine von der Mitgliederversammlung mit der Finanzverwaltung des Vereins beauftragt wird.
  \item Der Vorstand ist für die Vertretung des Vereins im Außenverhältnis verantwortlich.
  \item Der Vorstand wird von der Mitgliederversammlung gewählt und bleibt im Amt, bis die Mitgliederversammlung einen neuen Vorstand bestellt.
  \item Jedes Vorstandsmitglied ist allein vertretungsberechtigt.
  \item Der Vorstand ist beschlussfähig, wenn zwei Mitglieder des Vorstandes anwesendsind.
  \item Beschlüsse im Vorstand werden mit einfacher Mehrheit gefasst.
  \item Der Vorstand wird ermächtigt, Satzungsänderungen, die für eine Eintragung des Vereins oder eine Anerkennung als gemeinnützig auf gerichtliche oder behördliche Anregung erfolgen, zu beschließen.
  \item Mitglieder dürfen den Sitzungen des Vorstands beiwohnen, dürfen aber vom Vorstand für einen vertraulichen Abschnitt ausgeschlossen werden.
\end{enumerate}

\section{Ausschluss eines Mitglieds}
\begin{enumerate}
  \item Ein Mitglied kann durch Beschluss des Vorstandes ausgeschlossen werden, wenn es das Ansehen des Vereins schädigt, gegen den Verhaltenskodex des Vereins verstößt, oder wenn ein sonstiger wichtiger Grund vorliegt.
  \item Der Vorstand muss dem auszuschließenden Mitglied den Beschluss in Textform unter Angabe von Gründen mitteilen und ihm auf Verlangen eine Anhörung gewähren.
  \item Gegen den Beschluss des Vorstandes kann Widerspruch eingelegt werden. Der Widerspruch führt zur Überprüfung des Ausschlusses durch die Mitgliederversammlung. Bis zum Beschluss der Mitgliederversammlung ruht die Mitgliedschaft. Die einfache Mehrheit der Mitgliederversammlung muss den Ausschluss bestätigen.
\end{enumerate}

\section{Auflösung}
\begin{enumerate}
  \item Über die Auflösung des Vereins entscheidet die Mitgliederversammlung.
  \item Bei Auflösung oder Aufhebung des Vereins oder bei Wegfall steuerbegünstigter Zwecke fällt das Vermögen des Vereins an eine juristische Person des öffentlichen Rechts oder eine andere steuerbegünstigte Körperschaft zwecks Verwendung für den Vereinszweck.
\end{enumerate}

\end{document}

